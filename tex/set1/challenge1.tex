\documentclass[a4paper,10pt]{article}

\usepackage{a4wide}
\parindent 0em
\parskip 1em

\usepackage{algpseudocode, algorithm}

\begin{document}

\title{Challenge 1}
\date{}

\maketitle

\section{Problem} % (fold)
\label{sec:problem}

{\bf Convert hex to base64}

The string:

\verb+49276d206b696c6c696e6720796f757220627261696e206c696b65206120706f69736f6e6f7573206d757368726f6f6d+

Should produce:

\verb+SSdtIGtpbGxpbmcgeW91ciBicmFpbiBsaWtlIGEgcG9pc29ub3VzIG11c2hyb29t+

So go ahead and make that happen. You'll need to use this code for the rest of the exercises. 
% section problem (end)

\section{Solution} % (fold)
\label{sec:solution}

Hex represents a value in base 16 e.g. $f6d = f\cdot16^2 + 6\cdot16 + d$
where $a=10, b=11, ... f=15$.

Similarly, base64 represents a value in base 64, where the alphabet used
is $A=0, B=1, \dots, a=26, b=27, \dots, 0=52, 1=53, \dots, + = 62, / = 63$.

Since a single character of hex is 4 bits, and a character of base64 is 6 bits, 
we can see that three hex characters can be converted to two base64 characters:

\begin{align*}
 &h_2 \cdot 16^2 + h_1 \cdot 16 + h_0 \\
= &h_2 \cdot 16^2 + (h_1^{(1)}\cdot 4 + h_1^{(2)})\cdot 16 + h_0 \\
= &h_2 \cdot 256 + h_1^{(1)} \cdot 64 + h_1^{(2)}\cdot 16 + h_0 \\
= &(4h_2 + h_1^{(1)})\cdot64 + (16 h_1^{(2)} + h_0)
\end{align*}
where $h_1^{(1)}\cdot 4 + h_1^{(2)}$ is the base4 representation of $h_1$.

To see this is valid, we require that $4h_2 + h_1^{(1)} < 64$ and 
$16 h_1^{(2)} + h_0 < 64$.

This is clear since $h_2 \leq 15$ so $4h_2 \leq 60$ and $ h_1^{(1)} \leq 3$.

Similarly, $h_1^{(2)} \leq 3$, so $16 h_1^{(2)} \leq 48$ and $h_2 \leq 15$.
% section solution (end)


\end{document}
